\documentclass[12pt,answers]{exam}
\usepackage[utf8x]{inputenc}
\usepackage[english,brazil]{babel}	
\usepackage[margin=1in]{geometry}
\usepackage{amsmath,amssymb}
\usepackage{multicol}
\usepackage{graphicx}
\usepackage{lipsum}
\usepackage{listings}
\usepackage{color}

\definecolor{mygreen}{rgb}{0,0.6,0}
\definecolor{mygray}{rgb}{0.5,0.5,0.5}
\definecolor{mymauve}{rgb}{0.58,0,0.82}

\lstset{ %
  backgroundcolor=\color{white},   % choose the background color; you must add \usepackage{color} or \usepackage{xcolor}
  basicstyle=\footnotesize,        % the size of the fonts that are used for the code
  breakatwhitespace=false,         % sets if automatic breaks should only happen at whitespace
  breaklines=true,                 % sets automatic line breaking
  captionpos=b,                    % sets the caption-position to bottom
  commentstyle=\color{mygreen},    % comment style
  deletekeywords={...},            % if you want to delete keywords from the given language
  escapeinside={\%*}{*)},          % if you want to add LaTeX within your code
  extendedchars=true,              % lets you use non-ASCII characters; for 8-bits encodings only, does not work with UTF-8
  frame=single,                    % adds a frame around the code
  keepspaces=true,                 % keeps spaces in text, useful for keeping indentation of code (possibly needs columns=flexible)
  keywordstyle=\color{blue},       % keyword style
  language=make,                 % the language of the code
  morekeywords={*,...},            % if you want to add more keywords to the set
  numbers=left,                    % where to put the line-numbers; possible values are (none, left, right)
  numbersep=5pt,                   % how far the line-numbers are from the code
  numberstyle=\tiny\color{mygray}, % the style that is used for the line-numbers
  rulecolor=\color{black},         % if not set, the frame-color may be changed on line-breaks within not-black text (e.g. comments (green here))
  showspaces=false,                % show spaces everywhere adding particular underscores; it overrides 'showstringspaces'
  showstringspaces=false,          % underline spaces within strings only
  showtabs=false,                  % show tabs within strings adding particular underscores
  stepnumber=2,                    % the step between two line-numbers. If it's 1, each line will be numbered
  stringstyle=\color{mymauve},     % string literal style
  tabsize=2,                       % sets default tabsize to 2 spaces
  title=\lstname                   % show the filename of files included with \lstinputlisting; also try caption instead of title
}

                                  
\newcommand{\name}{José Pedro de Santana Neto}
\newcommand{\class}{Fundamentos de Redes de Computadores} %materia
\newcommand{\examnum}{Respostas do Experimento SMTP}  %Conteudo
\newcommand{\examdate}{\today}

\pagestyle{head}
\firstpageheader{}{\includegraphics[width=\textwidth]{./fga.jpg}}{}
\runningheader{}{Página \thepage\ de \numpages}{\examdate}
\runningheadrule


\usepackage[
pdfauthor={\name},
pdftitle={\class~-~\examnum},
pdfsubject={\examnum},
pdfkeywords={\class},
backref=true,
draft=false,
pdfstartview=FitH,
bookmarks=true,
bookmarksopen=false,
pdftoolbar=true,
colorlinks=true,
linkcolor=black,
urlcolor=black,
citecolor=black,%blue
pdftex,
bookmarks=true,
linktocpage=true,   % makes the page number as hyperlink in table of content
hyperindex=true,
unicode=false
]{hyperref}
\usepackage[all]{hypcap}

\begin{document}

\noindent
\begin{tabular*}{\textwidth}{l @{\extracolsep{\fill}} r @{\extracolsep{6pt}} l}
\textbf{\class} &&\\
\textbf{Nome: }\name & 
\textbf{Matricula:} 09/0119355 &\\
\textbf{Nome: }Hebert Douglas de Almeida Santos & 
\textbf{Matricula:} 10/0103979 &\\
\textbf{Nome: }André Bernardes Soares Guedes & 
\textbf{Matricula:} 12/0110181 &\\
\textbf{\examnum} & &\\
\textbf{\examdate} &&
\end{tabular*}\\
\rule[2ex]{\textwidth}{2pt}

%----------------------------Aqui começa a nota ------------------------------

\begin{questions}

\question \textbf{Descrever as principais modifcações realizadas nos arquivos /etc/postfx/main.cf e /etc/postfx/master.cf e os efeitos de cada modifcação. Anexar os arquivos no relatório}

As modificações no arquivo main.cf foram:
\begin{lstlisting}
myorigin=example.com
smtpd_banner = $myhostname ESMTP $mail_name
relayhost = smtp.yourisp.com
inet_interfaces = all
mynetworks_style = host
inet_protocols=ipv4
local_recipient_maps =
mydestination =
delay_warning_time = 4h
unknown_local_recipient_reject_code = 450
maximal_queue_lifetime = 7d
minimal_backoff_time = 1000s
maximal_backoff_time = 8000s
smtp_helo_timeout = 60s
smtpd_recipient_limit = 16
smtpd_soft_error_limit = 3
smtpd_hard_error_limit = 12
smtpd_helo_restrictions = permit_mynetworks, warn_if_reject reject_non_fqdn_hostname,
		reject_invalid_hostname, permit
smtpd_sender_restrictions = permit_mynetworks, warn_if_reject reject_non_fqdn_sender,
		reject_unknown_sender_domain, reject_unauth_pipelining, permit
smtpd_client_restrictions = reject_rbl_client sbl.spamhaus.org,
		reject_rbl_client blackholes.easynet.nl
smtpd_recipient_restrictions = reject_unauth_pipelining, permit_mynetworks,
		reject_non_fqdn_recipient, reject_unknown_recipient_domain,
		reject_unauth_destination, permit
smtpd_data_restrictions = reject_unauth_pipelining
smtpd_helo_required = yes
smtpd_delay_reject = yes
disable_vrfy_command = yes
alias_maps = hash:/etc/postfix/aliases
alias_database = hash:/etc/postfix/aliases
virtual_mailbox_base = /var/spool/mail/virtual
virtual_mailbox_maps = mysql:/etc/postfix/mysql_mailbox.cf
virtual_alias_maps = mysql:/etc/postfix/mysql_alias.cf
virtual_mailbox_domains = mysql:/etc/postfix/mysql_domains.cf
virtual_uid_maps = static:5000
virtual_gid_maps = static:5000
\end{lstlisting}

As modificações no arquivo master.cf foram:
\begin{lstlisting}
submission inet n       -       n       -       -       smtpd
  -o smtpd_sasl_auth_enable=yes
  -o smtpd_tls_auth_only=yes
  -o smtpd_client_restrictions=permit_sasl_authenticated,reject_unauth_destination,reject
  -o smtpd_sasl_security_options=noanonymous,noplaintext
  -o smtpd_sasl_tls_security_options=noanonymous
smtps     inet  n       -       -       -       -       smtpd
  -o smtpd_tls_wrappermode=yes
  -o smtpd_sasl_auth_enable=yes
  -o smtpd_tls_auth_only=yes
  -o smtpd_client_restrictions=permit_sasl_authenticated,reject
  -o smtpd_sasl_security_options=noanonymous,noplaintext
  -o smtpd_sasl_tls_security_options=noanonymous
\end{lstlisting}

\question \textbf{Qual a diferença entre o armazenamento de e-mails no formato mbox e maildir? Instale o maildir e descreva quais os procedimentos de instalação de um e outro caso.}

\input{questao2}

\question \textbf{Para que servem os esquemas de autenticação SASL/TLS. Caso tenha instalado, descreva qual foi a solução adotada (Dovecot ou outro) e explique os passos de instalação.}

  Os esquemas de autenticação SASL/TLS servem para apoderar a conexão tipicamente TCP por protocolos de segurança. 

O SASL é destinado ao uso de um conjunto de protocolos de criptografia e de primitivas iniciais para consolidar a negociação do cliente e o servidor. Esse processo é conhecido como $handshake$.

Já o TLS é uma subcamada entre a camada de aplicação e camada de transposte. TLS é uma API que substitui os procedimentos de chamada diretada da camada de transporte (TCP) por funções específicas e procoladas. A partir desse método que consolidado a conexão HTTPS que se embaseia em chaves de sessões.

Ambos protocolos não são compatíveis um com o outro.

\question \textbf{O que é a diferença entre os formatos RFC822 e MIME types defnido para e-mails? Explicá-los.}

\input{questao4}


\question \textbf{ Faça uma conexão telnet no servidor SMTP e envie um e-mail para qualquer usuário cadastrado utilizando os comandos do protocolo (HELO, MAIL FROM, RCPT TO, DATA, QUIT, etc.) Anotar os resultados dessa experiência.}

\begin{figure}[h]
     \centering
       \includegraphics[keepaspectratio=true,scale=0.49]{telnet.png}
     \caption{Enviando e-mail pelas diretivas do protocolo}
\end{figure}


\question \textbf{Faça uma conexão telnet no servidor POP e receba e-mails utilizando os comandos deste protocolo (USER, PASS, LIST, RETR e QUIT). Anotar os resultados dessa experiência.}

O servidor de e-mail configurado foi do tipo IMAP. Para operação de envio e recepção (feitos na plataforma $thunderbird$) de e-mails segue procedimentos e resultados:

\begin{figure}[h]
     \centering
       \includegraphics[keepaspectratio=true,scale=0.3]{send.png}
     \caption{Procedimento de envio de e-mail}
\end{figure}

\begin{figure}[h]
     \centering
       \includegraphics[keepaspectratio=true,scale=0.3]{confirmacao.png}
     \caption{Confirmacao de envio de e-mail}
\end{figure}

\begin{figure}[h]
     \centering
       \includegraphics[keepaspectratio=true,scale=0.49]{log.png}
     \caption{Log do Envio}
\end{figure}

\end{questions}
	

\end{document}
