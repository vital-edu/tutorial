\lstdefinestyle{base}{
  moredelim=**[is][\color{blue}]{|}{|},
}

\chapter{Migração de dados}
\begin{enumerate}
\item Pare o serviço do noosfero
\begin{lstlisting}
$ sudo service noosfero stop
\end{lstlisting}
\item Crie uma pasta para o backup  do banco de dados
\begin{lstlisting}[style=base]
$ mkdir -p |<nome da pasta de backup>|
\end{lstlisting}
\item Dê permissão total a todos os usuários para a pasta 
\begin{lstlisting}[style=base]
$ sudo chmod 777 |<nome da pasta de backup>|
\end{lstlisting}
\item Realize o backup do banco de dados
\begin{lstlisting}[style=base]
$ sudo -u postgres pg_dump |noosfero_database| > noosfero_bkp.sql
\end{lstlisting}

\textcolor{red}{{\scriptsize Obs: substitua  “noosfero\_database” pelo nome do banco de dados que se deseja realizar o backuop. Por padrão, o noosfero utiliza: }}

\textcolor{red}{{\scriptsize \textbf{noosfero\_test} para o ambiente de produção}}

\textcolor{red}{{\scriptsize \textbf{noosfero\_development} para o ambiente de desenvolvimento }}

\textcolor{red}{{\scriptsize \textbf{noosfero\_production} para o ambiente de produção.}}

\item Entre na página de backup criada no passo 2
\begin{lstlisting}[style=base]
$ cd |<nome da pasta de backup>|
\end{lstlisting}

\item Compacte o arquivo sql com o gzip
\begin{lstlisting}
$ gzip noosfero_bkp.sql
\end{lstlisting}

\item Realize o backup dos dados/arquivos do noosfero
\begin{lstlisting}
$ tar zcvf noosfero-data.tgz /var/lib/noosfero-data/
\end{lstlisting}

\item Envie os backups para a nova VM
\begin{lstlisting}[style=base]
$ scp -P22 noosfero-data.tgz noosfero_bkp.sql.gz |usuario|@|IP_de_destino|
\end{lstlisting}

\textcolor{red}{{\scriptsize Obs: não esqueça de substituir corretamente o usuário e o ip de destino corretamente.}}

\item Pare o serviço da nova VM, caso esteja em execução
\begin{lstlisting}
$ sudo service noosfero stop
\end{lstlisting}

\item Descompacte o “noosfero-data.tgz”
\begin{lstlisting}
$ tar -xzf noosfero-data.tgz -C /
\end{lstlisting}

\item Altere o dono na pasta
\begin{lstlisting}[style=base]
$ sudo chown -R |noosfero|. /var/lib/noosfero-data
\end{lstlisting}

\textcolor{red}{{\scriptsize Obs: altere “noosfero” caso o usuário do banco de dados da nova VM seja outro. }}

\item Descompacte o arquivo “noosfero\_bkp.sql.gz”
\begin{lstlisting}
$ gunzip noosfero_bkp.sql.gz
\end{lstlisting}

\item Derrube o database atual
\begin{lstlisting}
$ sudo -u postgres psql -c "DROP DATABASE noosfero_database"
\end{lstlisting}

\textcolor{red}{{\scriptsize  Obs: substitua  “noosfero\_database” pelo nome do banco de dados que se deseja derrubar. Por padrão, o noosfero utiliza: }}

\textcolor{red}{{\scriptsize	\textbf{noosfero\_test} para o ambiente de produção}}

\textcolor{red}{{\scriptsize	\textbf{noosfero\_development} para o ambiente de desenvolvimento}}

\textcolor{red}{{\scriptsize 	\textbf{noosfero\_production} para o ambiente de produção. }}

\item Crie o novo database
\begin{lstlisting}[style=base]
$ sudo -u postgres psql -c "CREATE DATABASE noosfero_database WITH OWNER = |noosfero|"
\end{lstlisting}

\textcolor{red}{{\scriptsize Obs: substitua  “noosfero\_database” pelo nome do banco de dados que se deseja criar. Por padrão, o noosfero utiliza:}}

\textcolor{red}{{\scriptsize	\textbf{noosfero\_test} para o ambiente de produção}}

\textcolor{red}{{\scriptsize	\textbf{noosfero\_development} para o ambiente de desenvolvimento}}

\textcolor{red}{{\scriptsize	\textbf{noosfero\_production} para o ambiente de produção.}}

\textcolor{red}{{\scriptsize Altere também "noosfero"  caso o usuário do banco de dados deva ser outro. }}

\end{enumerate}