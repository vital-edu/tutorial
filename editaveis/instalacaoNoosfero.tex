\lstset{language=sh, numbers=none, breaklines=true}

\lstdefinestyle{base}{
  moredelim=**[is][\color{blue}]{|}{|},
}

\chapter{Instalação e configuração do Noosfero }

O Vim é o editor de texto padrão que será utilizado nesse tutorial sempre que for necessário editar um arquivo. Caso deseje utilizar um outro editor de texto, substitua “vim” nos comandos desse tutorial pelo comando do editor de texto de sua preferência.

Todos os comandos desse tutorial devem ser executados em um terminal. Os comandos são precedidos por:

\begin{itemize}
\item \textbf{\$}: quando deve ser executado como usuário comum;
\item \textbf{\#}: quando deve ser executado como super-usuário(administrador, root).
\end{itemize}

\section{Instalando as dependências principais do Noosfero}

\begin{enumerate}[label=\alph*)]

\item Abra o arquivo sources do debian:
\begin{lstlisting}
$ sudo vim /etc/apt/sources.list
\end{lstlisting}

\item Adicione a seguinte linha ao final do arquivo:
\begin{lstlisting}
$ deb http://download.noosfero.org/debian/wheezy ./
\end{lstlisting}

\item Baixe a chave do repositório:
\begin{lstlisting}[numbers=none, breaklines=true]
$ wget -O - http://download.noosfero.org/debian/signing-key.asc | sudo apt-key add -
\end{lstlisting}

\item Atualize a lista de pacotes do debian:
\begin{lstlisting}
$ sudo apt-get update
\end{lstlisting}

\item Instale o postgresql:
\begin{lstlisting}
$ sudo apt-get install postgresql
\end{lstlisting}

\item Instale o noosfero:
\begin{lstlisting}
$ sudo apt-get install noosfero noosfero-apache
\end{lstlisting}

\end{enumerate}

\section{Instalando o PostFix e SASL}

\begin{enumerate}[label=\alph*)]

\item Instale o Postfix e SASL
\begin{lstlisting}
$ sudo aptitude install postfix libsasl2 ca-certificates libsasl2-modules
\end{lstlisting}

\item Realize a copia do certificado para pasta do PostFix
\begin{lstlisting}
$ cat /etc/ssl/certs/Thawte_Premium_Server_CA.pem | sudo tee -a /etc/postfix/cacert.pem
\end{lstlisting}

\item Abra o arquivo main.cf para habilitar o SASL com o editor de texto Vim
\begin{lstlisting}
$ sudo vim /etc/postfiz/main.cf
\end{lstlisting}

\item Procure a linha 'mydestination' e adicione o domínio do site a ser hospedado
\begin{lstlisting}[style=base]
mydestination = |<exemplo-dominio.com.br>|
\end{lstlisting}

\textcolor{red}{{\scriptsize  Obs: Subtitua <exemplo-dominio.com.br> removendo <> }}
\textcolor{red}{{\scriptsize  As informações já contidas devem ser mantidas, deve-se apenas acrescentar a URL desejada separada por vírgula. }}

\item Adicione as seguintes linhas de configuração no final do arquivo:

\begin{lstlisting}
relayhost = [smtp.gmail.com]:587
smtp_use_tls = yes
smtp_sasl_auth_enable = yes
smtp_sasl_password_maps = hash:/etc/postfix/sasl/sasl_
smtp_sasl_security_options = noanonymous
smtp_sasl_tls_security_options = noanonymous
smtp_tls_CAfile = /etc/postfix/cacert.pem
virtual_alias_maps = hash:/etc/postfix/virtual
\end{lstlisting}

\textcolor{red}{{\scriptsize Obs: O tutorial está especificado para o gmail, se estiver utilizando e-mail de outro domínio verifique suas configurações (SMTP e porta) e altere a linha relayhost.}}

\textcolor{red}{{\scriptsize Para realizar os próximos passos deste tutorial é necessário que já esteja disponível o e-mail e senha da organização que utilizará a 	plataforma noosfero. Este será o mecanismo de comunicação entre os usuários da plataforma e o administrador. }}

\item Crie a pasta SASL para armazenar o email e senha do e-mail
\begin{lstlisting}
$ sudo mkdir /etc/postfix/sasl
\end{lstlisting}

\item Crie o arquivo onde ficará armazenado o email e senha do email
\begin{lstlisting}
$ sudo touch /etc/postfix/sasl/sasl_passwd
\end{lstlisting}

\item Abra o arquivo com editor de texto Vim
\begin{lstlisting}
$ sudo vim /etc/postfix/sasl/sasl_passwd
\end{lstlisting}

\item Adicione email e senha no arquivo
\begin{lstlisting}[style=base]
	[smtp.gmail.com]:587 |<email da instituicao>|:|<senha de acesso ao email>|
\end{lstlisting}
\textcolor{red}{{\scriptsize Obs: substitua <email da instituição> e <senha de acesso ao email> removendo os: <> }}

\textcolor{red}{{\scriptsize O tutorial está especificado para o gmail, se estiver utilizando e-mail de outro domínio verifique suas configurações (SMTP e porta) e modifique a linha acima conforme as especificações. }}

\item Dê permissão apenas de leitura para o arquivo
\begin{lstlisting}
$ sudo chmod 400 /etc/postfix/sasl/sasl_passwd
\end{lstlisting}

\item Coloque o arquivo na base de dados do PostFix
\begin{lstlisting}
$ sudo postmap /etc/postfix/sasl/sasl_passwd
\end{lstlisting}

\item Abra o arquivo \emph{virtual} utilizando editor de texto Vim
\begin{lstlisting}
$ sudo vim /etc/postfix/virtual
\end{lstlisting}

\item Ao final do arquivo adicione as seguintes linhas de configuração
\begin{lstlisting}[style=base]
@|<dominio da maquina>|		|<email da organizacao>|
\end{lstlisting}
\textcolor{red}{{\scriptsize Obs: substitua <domínio da máquina> e <email da organização> removendo os: <> }}

\item Dê permissão de leitura para o arquivo
\begin{lstlisting}
$ sudo chmod 400 /etc/postfix/virtual
\end{lstlisting}

\item Coloque o arquivo na base de dados do PostFix
\begin{lstlisting}
$ sudo postmap /etc/postfix/virtual
\end{lstlisting}

\item Reinicie o PostFix para habilitar as alterações
\begin{lstlisting}
$ sudo service postfix restart
\end{lstlisting}

\end{enumerate}

\section{Configurando o backup do banco de dados}

\begin{enumerate}[label=\alph*)]

\item Abra arquivo “noosfero-backup.sh” com editor de texto Vim
\begin{lstlisting}
$ sudo vim /usr/bin/noosfero-backup.sh
\end{lstlisting}

\item Adicione as linhas seguintes para criação do script
\begin{lstlisting}[style=base]
#!/bin/bash
BKP_DIR=|<Local do Backup>|
date=$(date +backup_%Y_%m_%d_%H_%M)
mkdir $BKP_DIR/$date
chmod 777 $BKP_DIR

echo "Data do backup: " $(date +%Y/%m/%d-%H:%M)

cd $BKP_DIR

#Backup dos arquivos
tar zcvPf $BKP_DIR/$date/noosfero-
tar zcvPf $BKP_DIR/$date/noosfero-data.
tar zcvPf $BKP_DIR/$date/noosfero-etc.
sudo -u postgres pg_dump noosfero > $BKP_DIR/$date/noosfero_

scp -P21 -r $BKP_DIR/$date root@164.41.86.17:$BKP_DIR

#Apaga os backups com mais de 7 dias
cd $BKP_DIR
for i in `find $BKP_DIR/ -maxdepth 1 -type d -mtime +6 -print`; do rm -rf $i; done

echo "Local do backup: " $BKP_DIR/$date
\end{lstlisting}

\textcolor{red}{{\scriptsize Obs: Atenção para a linha  BKP\_DIR=<local do backup>, deve ser especificada a pasta onde o backup será realizado.}}

\end{enumerate}

\section{Agendando o backup para ser realizado todos os dias}

\begin{enumerate}[label=\alph*)]

\item Logue como root
\begin{lstlisting}
$ sudo su
\end{lstlisting}

\item Edite o agendador de tarefas cron
\begin{lstlisting}
# crontab -e
\end{lstlisting}

\item Adicione ao final do arquivo, a linha abaixo
\begin{lstlisting}
	30 0 * * * sh /usr/bin/noosfero-backup.sh
\end{lstlisting}

\item Saia do modo super usuário
\begin{lstlisting}
# exit
\end{lstlisting}

\item Caso a pasta especificada para o backup não exista, crie uma pasta
\begin{lstlisting}[style=base]
$ sudo mkdir -p |<pasta local para backup>|
\end{lstlisting}

\textcolor{red}{{\scriptsize Obs: substitua <pasta local para backup> pela pasta de backup de sua preferência, removendo os: <> }}

\end{enumerate}

\section{Realizando backup em uma máquina remota}

Os seguintes passos devem ser executados na máquina que esteja executando o Noosfero:

\begin{enumerate}[label=\alph*)]

\item Gere a chave ssh
\begin{lstlisting}
$ ssh-keygen -t dsa
\end{lstlisting}

\item Copie a chave pública gerada para a pasta \emph{Home}
\begin{lstlisting}
$ cp .ssh/id_dsa.pub .
\end{lstlisting}

\item Envie a chave ssh para a máquina que irá armazenar os backups
\begin{lstlisting}[style=base]
$ scp id_dsa.pub |<usuario da maquina destino>|@|<Ip da maquina destino>|:
\end{lstlisting}

Os seguintes passos devem ser executados na máquina que armazenará os backups:

\item Acesse a máquina remotamente através do comando
\begin{lstlisting}[style=base]
$ ssh |<usuario da maquina de destino>|@|<Ip da maquina de destino>|
\end{lstlisting}

\item Coloque a chave pública na lista de chaves autorizadas para que seja possível logar sem a necessidade de senha
\begin{lstlisting}
$ cat id_dsa.pub >> .ssh/authorized_keys
\end{lstlisting}

\item Crie o diretório onde o backup será realizado
\begin{lstlisting}[style=base]
mkdir -p |<diretorio do backup>|
\end{lstlisting}

\item Abra o arquivo “noosfero-backup.sh” com o editor de texto Vim
\begin{lstlisting}
$ sudo vim /usr/bin/noosfero-backup.sh
\end{lstlisting}

\item Adicione as linhas abaixo no script para apagar os backups antigos
\begin{lstlisting}
#!/bin/bash
BKP_DIR=/var/backups/portal

#Apaga os backups com mais de 7 dias
cd $BKP_DIR
for i in `find $BKP_DIR/ -maxdepth 1 -type d -mtime +3 -print`; do rm -rf $i; done
\end{lstlisting}
\item Agende o script de backup para executar todos os dias às 06:30 evitando a exclusão de algum backup

\begin{itemize}

\item Logue como root
\begin{lstlisting}
$ sudo su
\end{lstlisting}

\item Edite o agendador de tarefas cron
\begin{lstlisting}
# crontab -e
\end{lstlisting}

\item Adicione ao final do arquivo o agendamento
\begin{lstlisting}
# 30 6 * * * sh /usr/bin/noosfero-backup.sh
\end{lstlisting}

\item Sair do modo super usuário
\begin{lstlisting}
# exit
\end{lstlisting}
\end{itemize}
\end{enumerate}

\section{Instalando o Varnish}

\begin{enumerate}[label=\alph*)]
\item Instale o vanish
\begin{lstlisting}
$ apt-get install varnish
\end{lstlisting}

\item Instale o módulo apache RPAF
\begin{lstlisting}
$ sudo apt-get install libapache2-mod-rpaf
\end{lstlisting}
\end{enumerate}

\section{Configurando o Apache para ouvir a porta 8080}
\begin{enumerate}[label=\alph*)]

\item Abra o arquivo de configuração de portas do apache
\begin{lstlisting}
$ sudo vim /etc/apache2/ports.conf
\end{lstlisting}

\item Altere as seguintes linhas do arquivo
\begin{lstlisting}
NameVirtualHost *:80  para  NameVirtualHost *:8080
Listen 80  para  Listen 127.0.0.1:8080
\end{lstlisting}

\item Abra o arquivo sites-enabled do apache
\begin{lstlisting}
$ sudo vim /etc/apache2/sites-enabled/*
\end{lstlisting}

\item Altere a seguinte linha do arquivo
\begin{lstlisting}
<VirtualHost *:80>  para  <VirtualHost *:8080>
\end{lstlisting}

\item Reinicie o Apache
\begin{lstlisting}
$ sudo service apache2 restart
\end{lstlisting}

\end{enumerate}

\section{Configurando o Varnish}

\begin{enumerate}[label=\alph*)]
\item Abra o arquivo de configuração do varnish
\begin{lstlisting}
$ sudo vim /etc/default/varnish
\end{lstlisting}

\item Altere as seguintes linhas
\begin{lstlisting}
START=no  para  say START=yes
-a :6081  para  -a :80
\end{lstlisting}

\item Abra o arquivo dafault.vcl do Varnish
\begin{lstlisting}
$ sudo /etc/varnish/default.vcl
\end{lstlisting}

\item Inclua as seguintes linhas no final do arquivos na mesma ordem
\begin{lstlisting}
include "/etc/noosfero/varnish-noosfero.vcl";
include "/etc/noosfero/varnish-accept-language.vcl";
\end{lstlisting}

\item Reinicie o Varnish
\begin{lstlisting}
$ sudo service varnish restart
\end{lstlisting}

\end{enumerate}
\section{Ativando o log do Varnish}

\begin{enumerate}[label=\alph*)]
\item Abra o arquivo “varnishncsa”
\begin{lstlisting}
$ sudo vim /etc/default/varnishncsa
\end{lstlisting}

\item Descomente a seguinte linha
\begin{lstlisting}
VARNISHNCSA_ENABLED=1
\end{lstlisting}

\textcolor{red}{{\scriptsize Obs: O log do varnish será gravado em \emph{/var/log/varnish/varnishncsa.log}}}

\item Reinicie o varnish para que as mudanças tenham efeito:
\begin{lstlisting}
$ sudo service varnish restart
\end{lstlisting}
\end{enumerate}